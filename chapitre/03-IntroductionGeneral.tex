\chapter*{INTRODUCTION GÉNÉRALE}
\markboth{\MakeUppercase{INTRODUCTION GÉNÉRALE}}{}
%\addstarredchapter{INTRODUCTION GÉNÉRALES}
\addcontentsline{toc}{chapter}{INTRODUCTION GÉNÉRALE}
\adjustmtc
\thispagestyle{MyStyle}

\noindent Dans un contexte où les exigences en matière de sécurité, d'environnement industriel et de qualité deviennent toujours plus strictes, les entreprises doivent relever des défis complexes pour assurer leur conformité aux normes réglementaires tout en optimisant leurs processus. Cette réalité met en lumière la nécessité de solutions numériques innovantes, capables de répondre aux attentes des bureaux de conseil spécialisés dans ces domaines.

Dans ce cadre, notre projet vise à développer une application web intuitive et performante, dédiée à la gestion d'audits en ligne structurés par étapes. Conçue pour simplifier les processus d'audit, cette solution propose une interface utilisateur accessible et garantit la traçabilité ainsi que la sécurité des données. En offrant des outils modernes et adaptés, l'application ambitionne de faciliter la conformité aux standards professionnels tout en répondant aux besoins des entreprises en termes de simplicité et d'efficacité.

Ce projet s'inscrit ainsi dans une démarche d'innovation digitale, visant à transformer la gestion des audits pour répondre aux enjeux actuels de l'industrie, tout en assurant une expérience utilisateur fluide et sécurisée.

Ce document présente notre démarche complète de développement. Le premier chapitre expose le cadre général du projet et l'organisme d'accueil. Le deuxième chapitre détaille la spécification des besoins et l'analyse fonctionnelle. Le troisième chapitre présente l'étude technique et l'architecture retenue. Enfin, le quatrième chapitre décrit les différentes releases développées selon la méthodologie Scrum adoptée.
